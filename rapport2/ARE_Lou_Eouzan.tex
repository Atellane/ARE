% Created 2025-03-19 mer. 00:43
\documentclass[11pt]{article}
\usepackage{amsmath}
\usepackage{fontspec}
\usepackage[utf8]{inputenc}
\usepackage[T1]{fontenc}
\usepackage{graphicx}
\usepackage{longtable}
\usepackage{wrapfig}
\usepackage{rotating}
\usepackage[normalem]{ulem}
\usepackage{amsmath}
\usepackage{amssymb}
\usepackage{capt-of}
\usepackage{hyperref}
\usepackage{polyglossia}
\setdefaultlanguage{french}
\usepackage[margin=2cm]{geometry}
\usepackage{xcolor}
\definecolor{darkblue}{rgb}{0.0, 0.0, 0.5}
\usepackage{fontspec}
\setmainfont{Liberation Serif}
\usepackage{minted}
\usepackage{svg}
\svgsetup{inkscape=yes}
\setminted[python]{autogobble,breaklines,fontsize=\small}
\PassOptionsToPackage{draft}{transparent}  % Désactive le paquet problématique
\usepackage{fancyhdr}
\pagestyle{fancy}
\fancyhf{}
\fancyhead[L]{\small\authorname}
\renewcommand{\headrulewidth}{0pt}
\fancypagestyle{plain}{\fancyhf{}\fancyhead[L]{\small\authorname}}
\newcommand{\authorname}{Lou Eouzan}
\usepackage{hyperref}  % Doit être le DERNIER paquet chargé (sauf pour cleveref)
\author{Lou Eouzan}
\date{\today}
\title{Rapport d'analyse des données ARE 2025}
\hypersetup{
 pdfauthor={Lou Eouzan},
 pdftitle={Rapport d'analyse des données ARE 2025},
 pdfkeywords={},
 pdfsubject={},
 pdfcreator={Emacs 30.1 (Org mode 9.7.11)}, 
 pdflang={French}}
\begin{document}

\maketitle
\tableofcontents

\section{Introduction}
\label{sec:org6cf5ec0}

Lors de l'avant dernière séance nous avons réalisé une prise vidéo que nous avons du analyser à l'aide du logiciel kinovéa. Via le logiciel,nous
procèdons au tracking de la positions sur un axe x (horizontal) et un axe y (vertical) d'un point correspondant approximativement au centre de gravité
tout du long de la vidéo. L'objectif de ce rapport est d'abord d'analyser les mesures obtenues pour en déduires certaines données comme le rapport
entre le temps de mouvement et le temps d’immobilité, la vitesse d'escalade, le temps d'escalade et la hauteur du mur mais également de comparer
l'efficacité de ce mode de mesure avec celui utilisée lors du rapport 1 où nous avions mesuré l'accélérations linéaire sans g.
\section{Imports}
\label{sec:org13cef56}

\begin{minted}[]{python}
from matplotlib import pyplot as plt
import pandas as pd
import numpy as np
\end{minted}

Ce code importe des bibliothèques python dont nous avons besoin. La bibliothèque matplotlib nous permet de produire des graphiques pour avoir une
représentation visuelle des données. Enfin les bibliothèque numpy et pandas nous mettent à disposition des structures de données et des fonctions
associées pour stocker et analyser nos mesures.
\section{Ouverture du fichier et extraction des colonnes}
\label{sec:orgbbcdf20}

\begin{minted}[]{python}
data = pd.read_csv("data_traj_escalade_v1.5.csv", sep=";", decimal=",")
time = data["Time"]
x = data["Marqueur 1/0/X"]
y = data["Marqueur 1/0/Y"]
\end{minted}

La première ligne importe les données dans une variable data, ensuite les lignes suivantes stockent chaque colonne du dataframe obtenu en lisant les
données dans une variable correspondante. Les variables x et y stockent la position de notre point sur l'axe horizontal et vertical. Enfin la variable
time, elle contient les secondes correspondant au moment où chaque valeur de position a été mesurée.
\section{Obtention de l'allure de la montée et de la descente}
\label{sec:org022ec8a}

\begin{minted}[]{python}
print(np.where(data["Marqueur 1/0/Y"] == np.max(data["Marqueur 1/0/Y"])))
plt.clf()
plt.plot(x[:3433], y[:3433])
plt.xlabel("Position horizontale (en pixel)")
plt.ylabel("Position verticale (en pixel)")
plt.savefig("traj_montee.svg")
plt.clf()
plt.plot(x[3433:], y[3433:])
plt.xlabel("Position horizontale (en pixel)")
plt.ylabel("Position verticale (en pixel)")
plt.savefig("traj_descente.svg")
\end{minted}

\phantomsection
\label{org4aaaf22}
\begin{verbatim}
(array([3429, 3430, 3431, 3432]),)
\end{verbatim}


La première ligne renvoie la liste des index pour lequelles la positions sur l'axe vertical est la plus haute. Cette position la plus haute correspond
logiquement au haut du mur et donc à l'index de la fin de la montée. En l'occurence, le dernier index contenant la position la plus haute est l'index 3432.
Pour afficher la trajectoire lors de la montée et la descente on veut afficher la position sur l'axe x et l'axe y respectivement de l'index 0 à
l'index 3432 et de l'index 3433 à la fin du tableau. C'est ce dont ce charge les lignes suivantes.

\begin{center}
\includesvg[width=5in]{traj_montee}
\captionof{figure}{Tentative pour déterminer l'allure de la montée.}
\end{center}

\begin{center}
\includesvg[width=5in]{traj_descente}
\captionof{figure}{Tentative pour déterminer l'allure de la descente.}
\end{center}
\section{Tentative d'obtention de la vitesse}
\label{sec:org26de182}

\begin{minted}[]{python}
i = 0
vitesses_par_seconde = []
# 3402 = fin de la montée-30
while i <= 3402:
    time_a, x_a, y_a = data.iloc[i]
    time_b, x_b, y_b = data.iloc[i+30]
    dist_en_cm = (np.sqrt((x_b-x_a)**2+(y_a-y_b)**2))/1.6542
    vitesses_par_seconde.append(dist_en_cm/(time_b-time_a)) # vitesse en cm/s
    i+=30

print(np.mean(np.array(vitesses_par_seconde))*0.036) # vitesse moyenne en km/h
\end{minted}

\phantomsection
\label{orge7125cd}
\begin{verbatim}
0.2778923219981999
\end{verbatim}


Pour connaitre la vitesse en mètre par seconde, il est nécesssaire de convertir les distances en mètre, pour celà a été pris la mesure de la distance
entre mes pieds et le point de référence au début de la vidéo sur kinovéa. Cette distance s'est avéré être de 165,42 pixels correspondant après mesure
avec un mètre à 100 centimètre. Nous pouvons donc obtenir par un simple produit en croix que \(1 cm = 1 * 165,42/100\) pixels ce qui donne 1,6542
pixels. Nous devons donc diviser chaque distance par 1,6542 pour obtenir le résultat en cm. Le temps quand à lui est en secondes. La différence de
temps entre chaque mesures est, d'après une simple lecture des premières données d'environ 0,033 secondes, il faut environ 30 mesures \((1/0,033)\)
pour avancer d'une seconde. Les positions dont nous mesurons la distance correspondent donc à celle de la mesure x et de la mesure \(x+30\). Il suffit
ensuite de faire une moyenne des vitesses mesurées lors de la montée en centimètres par secondes. Nous tentons en ensuite de convertir la vitesse en
km/h, \(1 km = 100 000 cm\) donc on commence par diviser par 100 000, par ailleurs 1 heure = 3600 secondes donc on multiplie par 3600, on a donc
\(1 cm/s = (1/100 000)*3600 = 0,036 km/h\), il suffit donc de multiplier les cm/s par 0,036 pour obtenir la vitesse en km/h. La vitesse obtenue est donc
d'environ 0,28 km/h. Dans la mesure où, sur la vidéo nous pouvons voir que je suis assez lente et que par ailleurs je fais beaucoup de pause et
j'hésite à monter, cette vitesse moyenne pourrait être cohérente. Elle l'est en tout cas plus que celle trouvée lors du premier rapport qui mesurait
jusqu'à 10 000 m/s (\(= 10 000*3,6 = 36 000 km/h\) (on divise par 1000 pour que ce soit en km et on multiplie par 3600 pour que ce soit en heure ce qui
revient à faire \(x*(3600/1000)=x*3,6)\)) ce qui était totalement invraissemblable.

\begin{center}
\includesvg[width=5in]{speed}
\captionof{figure}{Vitesse mesurée au cours du temps lors de la montée.}
\end{center}
\section{Tentative d'obtention du temps d'escalade}
\label{sec:org63167ef}

\begin{minted}[]{python}
print(time[3432])
\end{minted}

\phantomsection
\label{org9530bb4}
\begin{verbatim}
125.648
\end{verbatim}


Pour mesurer le temps d'escalade il suffit tout simplement de regarder la valeur à l'index 3432 du tableau time correspondant au temps passé depuis le
début quand nous avons atteint le point le plus haut sur l'axe vertical, c'est à dire le haut du mur. Que ce soit pour ce rapport où le rapport
précédent, la mesure du temps était cohérente.
\section{Calcul du ratio temps de mouvements sur temps total}
\label{sec:org4738a14}

\begin{minted}[]{python}
def isclose(a, b, trsh):
  return abs(a-b)<trsh
trsh = 1
index_deb = time[(isclose(time, 11.23, trsh))]
index_fin = time[(isclose(time, 17.33, trsh))]
while len(index_deb) > 1:   
  trsh/=10
  index_deb = time[(isclose(time, 11.23, trsh))]
trsh = 1
while len(index_fin) > 1:
  trsh/=10
  index_fin = time[(isclose(time, 17.33, trsh))]
index_deb = index_deb.index[0]
index_fin = index_fin.index[0]
print(index_deb, index_fin)
\end{minted}

\phantomsection
\label{org059da6b}
\begin{verbatim}
337 520
\end{verbatim}


Pour trouver les temps d'immobilités, nous avons identifiés visuellement entre 11.23 secondes et 17.33 secondes un temps d'immobilité. Nous allons
donc chercher les index correspondant le plus à ces timecode puis mesurer la distance moyenne considérée comme parcourue sur cet interval et
considérer ensuite que toute distance parcourue supérieure à cette distance moyenne correspond à du mouvement. C'est cette recherche dont se charge ce
code en retrecissant le nombre d'élément autour de ces timecodes jusqu'à ne trouver plus que l'élément le plus proche.

\begin{minted}[]{python}
i = 0
dist_imo = []
while i < len(data[index_deb:(index_fin+1)])-30:
  time_a, x_a, y_a = data.iloc[i]
  time_b, x_b, y_b = data.iloc[i+30]
  dist_imo.append((np.sqrt((x_b-x_a)**2+(y_a-y_b)**2))/1.6542)
  i+=30
trsh = np.mean(np.array(dist_imo))
print(trsh)
\end{minted}

\phantomsection
\label{org7b096a2}
\begin{verbatim}
11.916358817123998
\end{verbatim}


Ce code ci se charge de calculer la distance moyenne parcouru sur cet interval pour pouvoir considéré que toute période où la distance parcouru est
supérieure ou égale à celle ci sera considérée comme une période de mouvement.

\begin{minted}[]{python}
i = 0
mouvement = []
while i <= (len(time)-30):
    time_a, x_a, y_a = data.iloc[i]
    time_b, x_b, y_b = data.iloc[i+30]
    dist_en_cm = (np.sqrt((x_b-x_a)**2+(y_a-y_b)**2))/1.6542
    if dist_en_cm > trsh:
        mouvement.append(True)
    else:
        mouvement.append(False)
    i+=30
mouvement_def = []
for j in mouvement:
    for k in range(30):
        mouvement_def.append(j)
if i < len(time):
    diff = len(time)-i
    for j in range(diff):
        mouvement_def.append(mouvement_def[-1])
print(f"{round(((len(time[(np.array(mouvement_def))])*0.033)-0.033)/(time[3432])*100, 2)} %")
\end{minted}

\phantomsection
\label{orge3a7293}
\begin{verbatim}
43.13 %
\end{verbatim}


Pour chaque intervalle de 1 seconde environ où on considère qu'il y a du mouvement, on ajoute True dans le tableau mouvement pour l'indiquer sinon on
ajoute false. Ensuite, nous créons un 2e tableau mais où chaque élément est présent 30 fois pour correspondre aux 30 index sautés pour faire nos
mesures sur un intervalle de 1 seconde. Enfin on vérifie que le tableau définitif contient bien le même nombre d'éléments que \textbf{time} et si ce n'est
pas le cas, on ajoute le dernier élément du tableau. C'est nécessaire car soit un tableau numpy quelconque arr et un tableau de booléen, arr[(cond)]
ne renvoie que les éléments de arr pour lequel cond est égal à True, en l'occurence correspondant aux moments où on est en mouvement. Chaque éléments
de ce tableau correspondants à 0,033 secondes on multiplie la taille du tableau obtenu par 0,033 puis on soustrait une fois 0,033 puisque au premier
élément du tableau 0 secondes se sont passées. On répète le même processus avec tous le tableau du temps. On fait ensuite le rapport entre les 2
chiffres obtenus puis on multiplie par 100 pour obtenir le pourcentage. Nous obtenons donc 33.43\%, ça parait peu dans la mesure où dans le dernier
rapport avait été mesuré 45.54\% de temps en mouvements alors même que je m'étais stoppée pendant de longue periode à plusieurs reprises. La méthode
avec accéléromètre semble donc plus efficace pour mesurer l'abscence ou non de mouvement.
\section{Tentative d'obtention de la hauteur du mur}
\label{sec:org3a69589}

\begin{minted}[]{python}
print((np.mean(np.array(vitesses_par_seconde))/100)*(time[3432]))
\end{minted}

\phantomsection
\label{org86c0126}
\begin{verbatim}
9.699059576230507
\end{verbatim}


Nous savons que \(vitesse = distance/temps\), ça signifie donc que \(distance = vitesse*temps\). Nous avons la vitesse en cm/s, il suffit de la diviser
par 100 pour obtenir la vitesse en mètre par secondes. Nous multiplions la vitesse par le temps total d'escalade et on obtient un résultat d'environ
9,6 mètres. Ce résultat parait réaliste. Lors du dernier rapport il n'avais pas été possible de calculer la hauteur du mur dans la mesure où la
vitesse n'était pas exploitable, l'utilisation de kinovea et du trackage est donc plus efficace pour cette mesure.
\section{Conclusion}
\label{sec:org9b627ec}

Dans ce rapport, nous avons mesuré la positions de mon centre de gravité dans un plan. De cette position et de son évolutions à travers le temps nous
avons obtenus la vitesse, le temps d'escalade, le ratio temps de mouvements sur temps total et la hauteur du mur. Cette méthode à globalement été plus
efficace et précise que celle de la mesure à l'aide de l'accéléromètre.
\end{document}
